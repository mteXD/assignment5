\documentclass{article}
\usepackage[utf8]{inputenc}
\usepackage[backend=biber,style=numeric,sorting=none]{biblatex}
\usepackage[english]{babel}
\usepackage{enumitem}
\usepackage{titlesec}
\usepackage{minted}
\usepackage{listings}
\usepackage{parskip}
\usepackage{changepage}
\usepackage{graphicx}
\usepackage{refcheck}
\usepackage{wrapfig}
\usepackage{caption}
\usepackage{subcaption}
\usepackage{float}
\usepackage{csquotes}
\usepackage{mwe}
\usepackage{hyperref, fancyhdr, abstract, times}
\usepackage{amsmath,amssymb,amsfonts}

\addbibresource{refs.bib} %Imports bibliography file

\pagestyle{fancy}
\fancyhf{}
\fancyhead[L]{\leftmark}
\fancyhead[R]{\thepage}
\renewcommand{\headrulewidth}{0.4pt}

% Title formatting
\titleformat{\section}
{\normalfont\Large\bfseries}
{\thesection}{1em}{}
\titleformat{\subsection}
{\normalfont\large\bfseries}
{\thesubsection}{1em}{}

% Title page information
\title{\LARGE \textbf{5th AAHPS Assignment}}

\author{
  \textbf{Arne Salobir} \\
  % \texttt{} \\
  \textbf{Matic Poženel} \\
  % \texttt{}
}

\date{\today}

\begin{document}
\maketitle

\begin{abstract}
  This report presents the 7 heuristic algorithms with test cases
  from the \textit{opfunu} package \cite{opfunu}.
\end{abstract}

\section{Random Local Search}

\subsection{Description}

Random Local Search (RLS) is a simple optimization algorithm that
iteratively explores the solution space by making random changes to
the current solution. It evaluates the new solution and accepts it if
it improves upon the current one, otherwise, it continues with the
current solution. This process is repeated until a stopping criterion
is met -- in our case, this is a maximum number of iterations, but it
could also be a satisfactory solution quality or a time limit.

\subsection{Results}

Running the random local search algorithm on the \textit{opfunu} library
produced the following results:

\begin{itemize}
  \item \textbf{Function: F12022}: \\
    Minimum value: $39623.0728669062$ \\
    Coordinates:
    $$
    -67.3879072802,  -7.5538226449,  49.0349491130,
    -13.0029941956,  -16.4569300873,  -49.1132197685,  -2.0007048233,
    59.5985707426,  -3.3996445467,  28.7209820316,  70.5346664686,
    17.0641938626,  57.6198825778,  22.9338097571,  31.4719475255,
    71.0572742801,  69.4658387565,  64.7949581750,  45.7295977516,
    -21.6395833786
    $$

  \item \textbf{Function: F22022}:
    Minimum value: $2651.8254613096$
    Coordinates: $$5.5267568486,  -65.6285678850,  -58.9350403892,
    47.4669411087,  38.9020679242,  -19.9806296654,  67.5978803374,
    -9.0027740271,  34.5072703351,  67.1072970921,  51.0354084107,
    9.8031547722,  -10.9559213644,  67.0999241302,  65.3757526535,
    35.7266077338,  -10.0309996504,  95.7943426482,  -24.1232534248,
    -37.7923466667$$

  \item \textbf{Function: F32022}:
    Minimum value: $601.5156231880$
    Coordinates: $$4.8207154926,  -7.2492674753,  -4.2450086507,
    -55.1225477636,  59.9889528340,  -11.6213636457,  -18.4178316096,
    -35.5323992981,  87.5495369439,  3.3867323599,  -16.5612824257,
    43.9333076248,  -28.5257426250,  41.9307799585,  -30.6918557968,
    35.8834417982,  58.5972380558,  -59.3145196626,  1.0796090376,
    42.2982275453$$

  \item \textbf{Function: F42022}:
    Minimum value: $1252.0517248194$
    Coordinates: $$67.3961366272,  -4.1039345960,  -72.0467664327,
    -75.1641317662,  -84.8490707244,  -40.2411054224,  37.5562963921,
    -10.3965582469,  -99.0039981135,  27.4092178420,  -53.0075600848,
    71.3849163533,  -33.0760366071,  50.6410752465,  -2.3145992260,
    53.7754561084,  -49.9141701012,  78.6654285543,  40.3206071441,
    -7.6396093225$$

  \item \textbf{Function: F52022}:
    Minimum value: $915.4243120388$
    Coordinates: $$-74.8117076562,  25.8430079614,  20.6517071331,
    60.1729487030,  61.7027926157,  -39.8621166198,  96.5551330059,
    57.3198516846,  -82.9403531786,  -45.6923215524,  -79.0131980076,
    13.2219627499,  56.3118766250,  -92.7265918671,  -64.5012217823,
    -22.6515943544,  6.8302549572,  26.3070063680,  -47.0576394144,
    73.9808880607$$

  \item \textbf{Function: F62022}:
    Minimum value: $1976965842.7894558907$
    Coordinates: $$48.0165271420,  -76.8614696108,  71.7577820304,
    52.0499078312,  -14.8964969869,  -87.7853531629,  -55.9340338672,
    95.6047538426,  -9.4667536431,  -67.8632338276,  -25.7908846816,
    28.8830990368,  3.1444738244,  -71.4872201020,  13.7604582730,
    34.7230828209,  68.2904185857,  45.9675411829,  1.4157603541,
    -27.5823505005$$

  \item \textbf{Function: F72022}:
    Minimum value: $4845.5967721097$
    Coordinates: $$-3.5935925257,  50.0126757956,  -96.6352144223,
    50.0605472636,  90.7191530398,  -17.9642832493,  46.6689782524,
    41.7077278391,  -78.8925965736,  88.2055810439,  33.2514597189,
    -93.5797511431,  12.5244326920,  -61.8936195431,  15.8827203345,
    -85.4472467043,  66.9454691269,  -74.8394019670,  97.9727373971,
    -14.6689170177$$

  \item \textbf{Function: F82022}:
    Minimum value: $451115640.0295203328$
    Coordinates: $$15.9575446943  89.8391075336  -24.0167407918
    -84.3759475247  9.8396655235  94.2546752394  55.1439016322
    30.7664782145  -31.1672048045  -9.2030667686  80.2814278674
    22.3098657820  -9.2410355445  14.6676451254  -48.4189152224
    -62.4299208760  -60.1589967261  -55.2346862259  83.9902753288
    46.1157707426$$

  \item \textbf{Function: F92022}:
    Minimum value: $4772.8426976289$
    Coordinates: $$86.6866269426  -29.6161453568  73.1998609070
    -81.0174456515  -45.3997717158  -20.4802838682  87.8342818638
    -50.1619372517  32.7422363637  -33.1722902721  7.3293846557
    77.1895269439  72.8281487685  -14.7237518617  74.7449935473
    14.1758949347  45.4963757882  -64.9227309879  -18.3750034078
    -90.4268675973$$

  \item \textbf{Function: F102022}:
    Minimum value: $3233.6947037421$
    Coordinates: $$-36.0986066209  -28.1816119127  72.0755209499
    -65.4300672057  -82.8375674291  66.7584709287  -28.1929501291
    -29.6888992170  -36.8163719073  -94.7549394151  24.9748138082
    -65.4035628905  17.5631691853  -77.6493840376  -84.5211004384
    -36.6647618389  51.9844024910  -77.2829521181  -12.3115218758
    -26.3126993450$$

  \item \textbf{Function: F112022}:
    Minimum value: $4418.3799883879$
    Coordinates: $$54.2548646183  -0.7209899378  10.7332477048
    33.5353526807  19.7186544196  30.0918543525  29.7189194759
    -15.6716503298  86.7212245313  -9.5992759103  68.2984319193
    -38.3342070841  48.9797408926  24.2616097932  32.0761884766
    12.1980105218  82.4866132809  -86.2198527137  27.0961328841  -99.2211222593$$

  \item \textbf{Function: F122022}:
    Minimum value: $3465.3264456618$
    Coordinates: $$-63.7985821997  53.6950300400  -81.1343253186
    -48.7154973795  -53.4178239683  88.1243415081  65.8941241811
    85.3185102737  57.6308477191  25.8517981743  -97.7756178377
    24.0702691153  84.4245995734  48.2687274029  29.7864160386
    -38.6240455568  98.7671174170  83.0775235924  86.1377494180  76.2793216154 \allowbreak
    $$

\end{itemize}

\section{Simulated Annealing}

\subsection{Description}

Simulated Annealing (SA) is a probabilistic optimization algorithm
that mimics the annealing process in metallurgy. It starts with a
high temperature, allowing the algorithm to explore the solution space
freely, and gradually cools down, reducing the probability of accepting
worse solutions. The algorithm accepts worse solutions with a certain
probability, which decreases as the temperature lowers. This helps
avoid local minima and allows the algorithm to explore a wider range of
solutions before converging to a local optimum.

The algorithm for simulated annealing can be described as follows:

Let $f(x)$ be the function to be minimized and defined over some domain
$D \subset \mathbb{R}^n$.

\begin{enumerate}
  \item \textit{Initialization}: Start with an initial solution $x_0
    \in D$, initial temperature $T_0 > 0$, and the iteration counter $k = 0$.
  \item \textit{Iteration}: create a candidate solution $x'$ in the
    neighborhood of the current solution $x_k$. Then compute the
    change in objective value: $\Delta f = f(x') - f(x_k)$. The
    candidate solution is accepted as the next state $x_{k+1}$ with
    the following probability:

    $$ P_{\text{accept}} =
    \begin{cases} 1 & \text{if} \ \Delta f\leq 0\\ \exp\left(
      -\frac{\Delta f}{T_{k}}\right) & \text{otherwise}
    \end{cases} $$

    Then, the temperature is updated according to the cooling
    schedule -- it could be as simple as $T_{k+1} = \alpha T_k$,
    where $\alpha < 1$ is a cooling factor. Finally, increment the
    iteration counter: $k = k + 1$.
  \item \textit{Termination}: Repeat the iteration step until a
    stopping criterion is met -- in our case, this is a maximum
    number of iterations.
\end{enumerate}

\subsection{Results}

\section{Gradient Descent Search}

\subsection{Description}

Gradient Descent Search (GDS) is an optimization algorithm that uses
the gradient of the objective function to find the direction in which
it must move to find a local minimum. It starts with an initial
solution and iteratively moves in the direction of the steepest
descent, which is determined by the negative gradient of the
objective function. The step size can be fixed or adaptive, and the
algorithm continues until it reaches a local minimum or a stopping
criterion is met.

The algorithm proceeds as follows:

Let $f(x)$ be the function to be minimized and defined over some domain
$D \subset \mathbb{R}^n$.

\begin{enumerate}
  \item \textit{Initialization}: Start with an initial solution $X_0
    \in D$ and set the iteration counter $k = 0$.
  \item \textit{Iteration}: Compute the gradient of the objective
    function at the current solution: $\nabla f(X_k)$. Then, update
    the current solution by moving in the direction of the steepest
    descent:

    $$ x_{k+1} = x_k - \alpha \nabla f(x_k) $$

    where $\alpha > 0$ is the step size. Finally, increment the
    iteration counter: $k = k + 1$.
  \item \textit{Termination}: Repeat the iteration step until a
    stopping criterion is met -- in our case, this is a maximum
    number of iterations.
\end{enumerate}

\subsection{Results}

\section{Best Descent Local Search}

\subsection{Description}

Best Descent Local Search (BDLS) is a local search algorithm that
starts with an initial solution and iteratively explores its
neighborhood. At each iteration, it evaluates $N$ neighboring solutions
and selects the one with the best objective function value. If this
neighboring solution improves upon the current solution, it becomes the
new current solution. The process continues until no neighboring
solution improves upon the current one, at which point the algorithm
terminates, having found a local optimum.

\begin{enumerate}
  \item \textit{Initialization}: Start with an initial solution $X_0$
    and set the iteration counter $k = 0$.
  \item \textit{Iteration}: Generate a set of $N$ neighboring solutions
    by adding random noise (amount of noise is controlled by our
    parameter $\delta$) to the current solution (current best value). For each
    neighboring solution, compute the objective function value. Select
    the neighboring solution with the best objective function value
    as the new current solution. If this neighboring solution improves
    upon the current solution, update the current solution to this new
    one. Finally, increment the iteration counter: $k = k + 1$.
  \item \textit{Termination}: Repeat the iteration step until a
    stopping criterion is met -- in our case, this is a maximum number
    of iterations.
\end{enumerate}

\subsection{Results}

\section{Guided Local Search}

\subsection{Description}

Guided Local Search (GLS) is an extension of local search algorithms
that incorporates additional information to guide the search process.
It uses a penalty mechanism to discourage the algorithm from revisiting
previously explored solutions. The algorithm maintains a set of
penalties for different features of the solution, and when a solution is
accepted, the penalties are updated based on the features of the
solution. This helps the algorithm explore new areas of the solution
space and avoid getting stuck in local minima.

The algorithm can be summarized as follows:
Let $f(x)$ be the function to be minimized and defined over some domain
$D \subset \mathbb{R}^n$.

\begin{enumerate}
  \item \textit{Initialization}: Start with an initial solution $X_0
    \in D$, a penalty vector $P_0$ initialized to zero, and set the
    iteration counter $k = 0$.
  \item \textit{Iteration}: Generate a set of neighboring solutions
    by adding random noise to the current solution. For each
    neighboring solution, compute the objective function value
    adjusted by the penalties: $f'(x) = f(x) + P(x)$. Select the
    neighboring solution with the best adjusted objective function
    value as the new current solution. Update the penalties based on
    the features of the new solution. The new penalty vector is
    computed using the so-called utility value, which is computed as
    $\text{max}\left( \frac{|\text{values}|}{1 + \text{penalty}}
    \right)$, where \textit{values} represents the vector of all
    current neighbors. Finally, increment the iteration counter: $k = k + 1$.
  \item \textit{Termination}: Repeat the iteration step until a
    stopping criterion is met -- in our case, this is a maximum
    number of iterations.
\end{enumerate}

\subsection{Results}

\section{Genetic Algorithm}

\subsection{Description}

Genetic Algorithm (GA) is a population-based optimization algorithm
that mimics the process of natural selection. It starts with a
population of candidate solutions, which are evaluated based on their
fitness. The algorithm selects the best solutions to form a new
generation through crossover and mutation operations. Crossover combines
two parent solutions to create offspring, while mutation introduces
random changes to the offspring. The process continues for a number of
generations, with the hope of evolving better solutions over time.

The algorithm can be summarized as follows:

Let $f(x)$ be the function to be minimized and defined over some domain
$D \subset \mathbb{R}^n$.

\begin{enumerate}
  \item \textit{Initialization}: Start with a population of $N$
    candidate solutions, each represented as a vector $X_i \in D$,
    where $i = 1, 2, \ldots, N$. Set the iteration counter $k = 0$.
  \item \textit{Iteration}: we perform the following steps for as many
    \textit{generations} (iterations) as we want to run the algorithm:
    \begin{enumerate}
      \item \textit{Evaluation}: Compute the fitness of each candidate
        solution in the population: $f(X_i)$ for $i = 1, 2, \ldots, N$.
      \item \textit{Selection}: Select a subset of the best candidate
        solutions based on their fitness; the size of the subset is
        called the \textit{tournament size}.
      \item \textit{Crossover}: Create offspring by combining pairs of
        selected candidate solutions. For each pair of parents, apply
        a crossover operator to create one or more offspring. In our
        case, we create a random mask, based on which we choose
        either elements from the first parent or the second parent,
        which leaves us with a new \textit{offspring} solution.
      \item \textit{Mutation}: Apply a mutation operator to the offspring
        to introduce random changes. The mutation operator can be as simple
        as flipping bits in a binary representation or adding random noise
        to continuous variables -- the latter is the case in our
        implementation, and we also control the amount of affected
        values with a \textit{mutation rate} parameter, which is
        $0.1$ by default.
      \item \textit{Replacement}: Replace the current population with the
        new generation of offspring. This can be done by replacing the
        entire population or by replacing only a portion of it.
    \end{enumerate}
  \item \textit{Termination}: Repeat the evaluation, selection,
    crossover, mutation, and replacement steps until we reach the
    stopping criterion -- for the genetic algorithm, this is almost
    always a maximum number of generations.
\end{enumerate}

\subsection{Results}

\section{Walrus Optimization}

\subsection{Description}

Walrus Optimization (WaO) is a nature-inspired optimization algorithm
that simulates the behavior of walruses in their natural habitat.
It mimics their social behavior and foraging strategies to explore the
solution space effectively -- walruses often congregate around dominant
or more experienced individuals in order to find better food sources.

Similarly, the algorithm starts with a population of walruses, each
representing a candidate solution. The algorithm evaluates the
fitness of each walrus and selects the best ones to form a new
generation. All other walruses are now attracted to the best walrus,
which represent the most promising solution. However, the algorithm
also allows walruses to have random movements (that is, not in the
direction of the best walrus) to maintain diversity in the population
and avoid premature convergence to local optima. The walruses move
through the solution space, exploring new areas and updating their
positions based on their fitness. The algorithm continues until a
stopping criterion is met, such as a maximum number of iterations or
a satisfactory solution quality.

Let's summarize the key steps of the Walrus Optimization algorithm:

\begin{enumerate}
  \item \textbf{Exploration phase}: we first calculate the new
    position ${X}_{i}^{{P}_{1}}$ for the $i$-th walrus. For the $j$-th dimension
    of this position, we use the following equation:
    $$
    {x}_{i,j}^{{P}_{1 }} = {x}_{i,j}+{rand}_{i,j} \cdot \left( SW_j -
    I_{i,j} \cdot x_{i,j} \right)
    $$

    We only update the best position if the objective value
    $F_i^{P_1}$ is an improvement
    compared to the previous objective value:

    $$
    {X}_{i}=\left\{
      \begin{array}{ll}{X}_{i}^{{P}_{1}}, &
        {F}_{i}^{{P}_{1}}<{F}_{i},\\ {X}_{i}, & else,
      \end{array}
      \right.
      $$

    \item \textbf{Migration phase}:

      we
  \end{enumerate}

  \subsection{Results}

  \printbibliography[title={References}]

  \end{document}
